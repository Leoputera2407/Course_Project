\documentclass[11pt]{article}
 \usepackage{minted}

\usepackage{amsmath}
\usepackage{xcolor}
 
 
\newcommand{\comment}[1]{}
 
\usepackage[a4paper]{geometry}
\usepackage[myheadings]{fullpage}
\usepackage{fancyhdr}
%\usepackage{fourier} % this is for the font -- see whether this is nicer than default
\usepackage[english]{babel}

\newcommand{\HRule}[1]{\rule{\linewidth}{#1}}
\setcounter{tocdepth}{5}
\setcounter{secnumdepth}{5}


\def\E{{\rm E}}
\def\Var{{\rm Var}}
\def\Cov{{\rm Cov}}


\usepackage{fancyvrb}

\author{Wilson Jusuf, 404997407}


\begin{document}

\title{ \normalsize \textsc{Spring 2019 CS118}
		\\ [2.0cm]
		\HRule{0.5pt} \\
		\LARGE \textbf{Project 1 - BSD Sockets Webserver}
		\HRule{2pt} \\ [0.5cm]
		\normalsize \today \vspace*{5\baselineskip}}

\date{}

\author{
        Wilson Jusuf\\
		Student ID: 404997407\\
		University of California, Los Angeles\\\\
        Hanif Leoputera Lim\\
		Student ID: 504971751\\
		University of California, Los Angeles}

\maketitle
\newpage

\section{high-level description of server}

First, we call the typical BSD system calls to initialize the server:

\begin{minted}[ breaklines, breakanywhere, tabsize=1, frame=single]{c}
     sockfd = socket(AF_INET, SOCK_STREAM, 0);
     setsockopt(sockfd, SOL_SOCKET, SO_REUSEADDR, &option, sizeof(option));
     bind(........);
\end{minted}
Then, we start to accept HTTP connections one by one in a while loop. 
The following pseudocode describes how we process an individual HTTP request:

%\begin{minted}[ breaklines, tabsize=1, frame=single, fontsize=10pt]{c}
\begin{minted}[ breaklines, breakanywhere, tabsize=1, frame=single]{c}
    // 1. read into a buffer.
    read(readbuf);


    // 2. parse in http request to struct and check if http request was in an invalid format.
    parse_success = parse_http_request(&http_req_struct)      
    if (!parse_success) 
        respond_with_status(400) // return 400 bad request


    // 3. check if method is a GET
    if (http_req_struct.method != "GET")
        respond_with_status(405) // return 405 method not allowed
    

    // 4. try to open the file. 
    fildes = open_file(http_req_struct.endpoint_resource);
    if (fildes is error) 
        write_fail(/* based on errno value. ENOENT = 404, EACCES = 403 */)
        

    // 5. check if mimetype is supported. If not, return error. If not, return error
    if(!mimetype_is_supported(http_req_struct.endpoint_resource))
        write_fail(415); // 415 media not supported


    // 6. get file length for content_length header. See if this fails.
    content_len = get_filelength(fildes); // uses stat syscall
    if(content_len error)
        write_fail(500) // 500 Internal Server Error
        

    // 7. craft our response header, without data. send it.
    get_200_OK_response_header(res_buff);
    write(connfd, res_buff);


    // 8. send file data to receiver.
    while (file reading still has data) {
            write(connfd, data) // send data to client.
    }


    // 9. Finally, close resources and connections. NOTE: they are also closed when we return early as well.
    close(connfd);
    close(fildes);
\end{minted}

Notes:
- we actually use \texttt{writev()} system call to issue both header + data writes to receiver to minimize syscall overhead.


\section{Difficulties faced}
The difficulty lied mostly on understanding the BSD abstractions on managing a webserver.
After reading the man pages, it seemed clear that the abstractions were almost like opening files and file descriptors, as well
as their associated system calls like \texttt{read()} and \texttt{write()}. It was actually simple.

\section{Manual - how to compile and run}
\begin{minted}[breaklines, breakanywhere, tabsize=1, frame=single]{bash} 
# first, make the webserver executable:
$ make

# then run default:
$ ./webserver  # this runs on port 7900 by default

# or, by specifying port option,
$ ./webserver -p 9500 # run on port 9500
\end{minted}

\section{Sample outputs and explanation}

PART A:
Upon receiving a request, the request header is outputted
\begin{minted}[breakanywhere, breaklines, frame=single]{text}
GET /test.txt HTTP/1.1
Host: localhost:7900
Connection: keep-alive
Upgrade-Insecure-Requests: 1
User-Agent: Mozilla/5.0 (Macintosh; Intel Mac OS X 10_12_6) AppleWebKit/537.36 (KHTML, like Gecko) Chrome/73.0.3683.103 Safari/537.36
Accept: text/html,application/xhtml+xml,application/xml;q=0.9,image/webp,image/apng,*/*;q=0.8,application/signed-exchange;v=b3
Accept-Encoding: gzip, deflate, br
Accept-Language: en-US,en;q=0.9
\end{minted}
Definitions are included for the headers above:
\begin{itemize}
  \item \texttt{GET}: This line has the request method (GET, POST, etc..), the path to the resource and the protocol type and version.
  \item \texttt{HOST}: The server's name -- in our case it's localhost, running on port 7900
  \item \texttt{Connection}: This specify the requested lifetime of the connection
  \item \texttt{User-agent}: This tells the server client's application type.In our case, it's a Mozilla browser, running on a Mac, which is powered by Intel.
  \item \texttt{Accept}: Describe what media/content client accept. Client accepts html, xml, webp and apng, in our case.
  \item \texttt{Accept-Encoding}: Specifies the encoding of the media accepted by client.
  \item\texttt{Accept-Language}: Specifies the accepted languages of the returned media.
\end{itemize}

PART B - requesting test.txt
\begin{minted}[breakanywhere, breaklines, frame=single]{text}
GET /test.txt HTTP/1.1
Host: localhost:7900
Connection: keep-alive
Upgrade-Insecure-Requests: 1
User-Agent: Mozilla/5.0 (Macintosh; Intel Mac OS X 10_12_6) AppleWebKit/537.36 (KHTML, like Gecko) Chrome/73.0.3683.103 Safari/537.36
Accept: text/html,application/xhtml+xml,application/xml;q=0.9,image/webp,image/apng,*/*;q=0.8,application/signed-exchange;v=b3
Accept-Encoding: gzip, deflate, br
Accept-Language: en-US,en;q=0.9
\end{minted}
The client issued a GET request for test.txt in the server. Upon checking that 
\begin{itemize}
    \item the requested resource's mimetype is a valid mimetype,
    \item the resource exists in the server, 
    \item the server has access rights to the resource, 
\end{itemize}

the server then sends it back to the user with the following:

\begin{minted}[breakanywhere, breaklines, frame=single]{text}
HTTP/1.1 200 OK
Date: Sat, 27 Apr 2019 04:07:28 UTC
Content-Type: text/plain; charset=utf8
Content-Length: 11

(.... data here ....)
\end{minted}

The first line shows the protocol type and version (HTTP/1.1) and the response code "200 OK", implying that the mimetype is valid and resource is found.
\begin{itemize}
  \item \texttt{Date}: specify the time the server sent the resource to the client.
  \item \texttt{Content-Type}: tells the client the type of file being returned
  \item \texttt{Content-Length}: the total size of the file sent.
\end{itemize}

The content of test.txt was diff-ed and is identical to the server's.

Similarly, when a client requests test.gif with the following:
\begin{minted}[breakanywhere, breaklines, frame=single]{text}
GET /test.gif HTTP/1.1
Host: localhost:7900
Connection: keep-alive
Upgrade-Insecure-Requests: 1
User-Agent: Mozilla/5.0 (Macintosh; Intel Mac OS X 10_12_6) AppleWebKit/537.36 (KHTML, like Gecko) Chrome/73.0.3683.103 Safari/537.36
Accept: text/html,application/xhtml+xml,application/xml;q=0.9,image/webp,image/apng,*/*;q=0.8,application/signed-exchange;v=b3
Accept-Encoding: gzip, deflate, br
Accept-Language: en-US,en;q=0.9
\end{minted}

The server sends back test.gif along with the following header:
\begin{minted}[breakanywhere, breaklines, frame=single]{text}
HTTP/1.1 200 OK
Date: Sat, 27 Apr 2019 04:34:24 UTC
Content-Type: image/gif
Content-Length: 121606

(... gif here...)
\end{minted}

\end{document}
